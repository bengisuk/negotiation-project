\todo{a conclusion in which you summarize your experience as a team with regards to building the negotiating agent and discuss what extensions are required to use your agent in real-life negotiations to support (or even take over) negotiations performed by humans.} \\ 

\todo{Het lijkt me hier op zijn plaats om het BOA framework te bekritiseren zoals het nu is. Dat je sommige componenten niet makkelijk kan bereiden (BS/AS) is echt heel onhandig...} \\

\subsection{Concluding: Future Perspectives}
In the proces of optimizing the agent to reach the best possible outcome of the negotiation session, it turned out to be hard to find a deal that is Pareto optimal. Another way of negotiating is by using a mediator, which objectively mediates between two or more agents. The agents which take part of the negotiation all get the same bid from the mediator, which the agents can accept or reject. This means that the \textit{bidding} actuator is now canceled, the only actuator now is \textit{accepting/rejecting}.\\

While agents can only reject or accept a bid of the mediator, so seeing no bids of the opponents, it is very hard to make a preference profile of the opponents. Therefore, it is most probable that agents will focus on their own utility. Besides that, agents won't care much about the utility of the other agent as long as they got a good utility themselves.  \\

Since the mediator has both preference profiles, it is able to calculate all the bids on the Pareto frontier including the \textit{Nash-point} and the \textit{Kalai-Smorodinsky solution}. An objective mediator can walk over the Pareto Optimal Frontier, offerings bids until both agents accept. An objective negotiation should end in the Nash-point or Kalai-Smorodinsky point, however, since we are at the optimal frontier, the first bid which is accepted by both parties will be the outcome. \\

This way of negotiating will compute an optimal outcome, however in a restricted optimal outcome space. There are less possibilities for strong negotiating agents to influence and analyze the opponent, so they could get a worse outcome. Especially in bidding spaces which are poorly distributed in favor of one agent, the agent will probably get a worse outcome.

\subsubsection{Other capabilities}
Extending agents with more communication abilities would probably not increase the performance significantly. They could
communicate about certain issues, for example if they would concede or stay. However, it will never be sure when an agent is telling the truth or using a communication strategy. \\

Communication that however could be useful, is information about what agents do not want. After communicating this, agents could neglect certain offers to shrink the negotiation space. This could save time, and could be useful when there is a time constraint or a discount factor. \\

However, another way to look at this is that the domain could be specified better at the start of the negotiation, by filtering the issues which never would be accepted by the opponent beforehand. For example, when buying a car, one could have a default domain where all possible values for all possible cars are specified, which is 
downloaded to your pocket negotiator. Then it would be useful to remove all the cars without certain features, so which are not matching your preferences, beforehand. \\

There are also other possibilities. Considering humans when negotiating, they are able to offer issues outside the negotiation domain, to convince the opponent to get this deal. Well known examples are money or, in the case of cars, extra tires. In the current Genius environment it is impossible to offer new issues in a negotiation session, maybe this could be a feature in the future when negotiators are able to rate unknown issues or when they can ask for input of a human. 

\subsection{Conclusions}

For this project we performed an in-depth study in the world of automated negotiation. After reading a large amount of literature written about the subject we have designed and implemented our own agent using the BOA framework and \texttt{Genius} environment. It became clear to us that negotiation is not as straight forward as it might seem at first glance. Finding outcomes that are a win-win situation for both parties can be hard. The complexity of the BOA components are expressed in the amount of different parameters, depending on a lot of factors. These factors are optimized in the testing phase.  \\

Our agent, the BOAconstructor, combines multiple existing ideas.


