\todo{a conclusion in which you summarize your experience as a team with regards to building the negotiating agent and discuss what extensions are required to use your agent in real-life negotiations to support (or even take over) negotiations performed by humans.} \\ 

\todo{Het lijkt me hier op zijn plaats om het BOA framework te bekritiseren zoals het nu is. Dat je sommige componenten niet makkelijk kan bereiden (BS/AS) is echt heel onhandig...} \\

\subsection{Concluding: Future Perspectives}
In the proces of optimizing the agent to reach the best possible outcome of the negotiation session, it turned out to be hard to find a deal that is Pareto optimal. Another way of negotiating is by using a mediator, which objectively mediates between two or more agents. The agents which take part of the negotiation all get the same bid from the mediator, which the agents can accept or reject. This means that the \textit{bidding} actuator is now canceled, the only actuator now is \textit{accepting/rejecting}.\\

While agents can only reject or accept a bid of the mediator, so seeing no bids of the opponents, it is very hard to make a preference profile of the opponents. Therefore, it is most probable that agents will focus on their own utility. Besides that, agents won't care much about the utility of the other agent as long as they got a good utility themselves.  \\

Since the mediator has both preference profiles, it is able to calculate all the bids on the Pareto frontier including the \textit{Nash-point} and the \textit{Kalai-Smorodinsky solution}. An objective mediator can walk over the Pareto Optimal Frontier, offerings bids until both agents accept. An objective negotiation should end in the Nash-point or Kalai-Smorodinsky point, however, since we are at the optimal frontier, the first bid which is accepted by both parties will be the outcome. \\

This way of negotiating will compute an optimal outcome, however in a restricted optimal outcome space. There are less possibilities for strong negotiating agents to influence and analyze the opponent, so they could get a worse outcome. Especially in bidding spaces which are poorly distributed in favor of one agent, the agent will probably get a worse outcome.

\subsubsection{Other capabilities}


This method would definitely 
We think that extending agents with more communication abilities would not 
Nash-points have the characteristic that it is the point which in total optimizes the utility for every agent. However, the Nash-point is not necessarily the optimal point for every individual agent. Therefore, it is dependent on the order of bids what the outcome of the negotiation will be. 
increase the perfromancy very much. When automated agents communicate,
you would need to define a protocol so they can understand eachother.
Agents could communicate things like: if you do this, I will concede here, etc...
However, using the current method of exchanging bids, this already becomes 
appararent. The advantage of working with bids is agents are sure and that they are final,
when communicating, agents can lie and take advantage of each other. 
We think that this will lead to no communication at all because of this danger.
One type of communication that would be useful is maybe that agents
tell eachother what bids they deem unacceptable. 
After communicating this, the agents will never offer these bids because they know
their opponent will never accept these. This way, 
agents will know what bids are unacceptable for the opponent, and this way
the domain size can be reduced. 
However, another way to look at this is that the domain is not specified well for the 
negotation if this is the case. Before you negotatiate, you need to specify the issues
and values for these issues, and if you include values that are not acceptable to 
yourself as negotatior, this is useless. However, one can imagine cases where
there is a default domain. For example, when buying a car, one could have a default domain
where all possible values for all possible cars are specified, and that this domain is 
downloaded to your pocket negotatior. Then this would be very useful,
to make negotations much faster.



\subsection{Conclusions}

For this project we performed an in-depth study in the world of automated negotiation. After reading a large amount of literature written about the subject we have designed and implemented our own agent using the BOA framework and \texttt{Genius} environment. It became clear to us that negotiation is not as straight forward as it might seem at first glance. Finding outcomes that are a win-win situation for both parties can be hard. The complexity of the BOA components are expressed in the amount of different parameters, depending on a lot of factors. These factors are optimized in the testing phase.  \\

Our agent, the BOAconstructor, combines multiple existing ideas.


