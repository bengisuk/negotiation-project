

For this project we performed an in-depth study in the world of automated negotiation. After reading a large amount of literature written about the subject we have designed and implemented our own agent using the BOA framework and \texttt{Genius} environment. It became clear to us that negotiation is not as straight forward as it might seem at first glance. Reaching an agreement that is a win-win situation for both parties can be quite hard.

Our agent, the BOAconstructor, combines multiple existing ideas.
However, we tried to implement the tit-for-tat strategy which was very hard to get to work properly, and we have also introduced multiple phases that introduced a lot of parameters. Our resulting agent is very complex, and because of this we had many difficulties optimizing the parameters of each of our components because of the high dimensional parameter search space.
Also because of the complexity there were a lot of bugs, which limited our testing time considerably. Because of this, our agent is clearly performing suboptimal. In the tests we have seen that our acceptance strategy in combination with the third phase of our bidding strategy almost always leads to acceptance, which is apparently a tad too lenient when facing most other agents. Furthermore, we found that many settings we tried for phase two, led to a rapid concession compared to our opponent.  Therefore, we have decided to disable these concessions in final tests. Sadly, we did not have time to optimize these parameters further.

We conclude that in automated negotiation, starting out with a simple agent and 
continually improving this agent, would have been a better strategy to develop agents. 
We clearly started out with a too complex agent before optimizing any parameters.
However, we think that separating a bidding strategy into phases can lead to 
better results than just using a single bidding strategy. By adding parameters
to each individual phase, we can explore the space of strategies even more
than just using a BOA agent.

Our mistakes might be in part because of the too large separation between members of our project group.
Each of us often worked on our 'own' component, and because of this, most of us made components too complex as there was no overview. Furthermore, we had some practical issues with version control, from which we learned a lot.
Finally, we have really enjoyed this challenge, although we aren't entirely satisfied with 
our results. We wish we could spend more time on this project looking back at our mistakes,
since we think our agent can be improved a lot.
 
\section{Future Perspectives}
\subsection{Improving BOA constructor}
During the development of the BOAconstructor agent, we found that there are features that could improve the results, but we didn't have time to implement. This happened for each of the BOA components.

The Bidding Strategy phase 2 has too many parameters which are not correctly optimized.
The agent could be considerable improved by this optimization. Furthermore, 
the last phase is also not fully optimized, sometimes conceding too much.

The Opponent Model now still uses default frequency modeling for adjusting the issue weights. Estimating the issue weights using a system similar to the one used to estimate the value weights might lead to improved results. Another method to potentially improve the results is to cross check whether the learn rate used doesn't brake the estimates for previous offers. In other words: check if after the value iteration the estimates of all previous bids are still within range of the expected utility. 

Our Acceptance Strategy uses many acceptance conditions which in some cases overlap. We can imagine that we could leave out one or two of these conditions. However, our results are promising, as we seem to get the highest utility when we use our own acceptance strategy. This is not surprising though, since we take a lot of cases into account. This makes our acceptance strategy really complex, and by optimizing the parameters and refactoring the code we could possibly reduce its complexity. We suggest to test each condition and its parameters separately, in order to find out if they actually improve or worsen our results.

\subsection{Future of BOA constructor for real-life negotations}
Before our agent -- or most negotiation agents -- can be used by humans in real negotation 
session, we think a lot has to be done. People probably want more control over the 
negotiation process of the agent, and specifically in understandable terms, e.g. no hard but fuzzy parameters.
Also it is important for people to understand what it is doing. 
Tutorials to see that the agent is performing well would be important. 
This might be best realized by letting people negotiate against the agent. 
This could illustrate the different tactics of the agent and give an idea of it's settings.

\subsection{Using a mediator}
In the process of optimizing the agent to reach the best possible outcome of the negotiation session, it turned out to be hard to find a deal that is Pareto optimal. Another way of negotiating is by using a mediator, which objectively mediates between two or more agents. The agents which take part of the negotiation all get the same bid from the mediator, which the agents can accept or reject. This means that the \textit{bidding} actuator is now canceled, the only actuator now is \textit{accepting/rejecting}.

Since agents can only reject or accept a bid of the mediator, so seeing no bids of the opponents, it is very hard to make a preference profile of the opponents. Therefore, it is most probable that agents will focus on their own utility. Besides that, agents won't care much about the utility of the other agent as long as they get a good utility themselves. 

Since the mediator has both preference profiles, it is able to calculate all the bids on the Pareto frontier including the \emph{Nash-point} and the \emph{Kalai-Smorodinsky solution}. An objective mediator can walk over the Pareto Optimal Frontier, offerings bids until both agents accept. An objective negotiation should end in the Nash-point or Kalai-Smorodinsky point. However, since we are at the optimal frontier, the first bid which is accepted by both parties will be the outcome.

This way of negotiating will compute an optimal outcome, however in a restricted optimal outcome space. There are less possibilities for strong negotiating agents to influence and analyze the opponent, so they could get a worse outcome. Especially in bidding spaces which are poorly distributed in favor of one agent, the agent will probably get a worse outcome.

So in conclusion, we think that it can be useful if both parties want to reach an optimal outcome.
However, if one of the negotiators wants to exploit the opponent, or just negotiate, a mediator
is not the solution, because it hardly allows for any real negotiation.

\subsection{Other possibilities to improve automated agents}
Extending agents with more communication abilities would probably not increase the performance significantly. They could
communicate about certain issues, for example if they would concede or stay. However, it will never be sure when an agent is telling the truth or using a communication strategy. 

Communication that however could be useful, is communication about what agents do not want. After communicating this, agents could neglect certain offers to shrink the negotiation space. This could save time, and could be useful when there is a time constraint or a discount factor. 

However, another way to look at this is that the domain could be specified better at the start of the negotiation, by filtering the issues which never would be accepted by the opponent beforehand. For example, when buying a car, one could have a default domain where all possible values for all possible cars are specified, which is 
downloaded to your pocket negotiator. Then it would be useful to remove all the cars without certain features, so which are not matching your preferences, beforehand. 

There are also other possibilities. Considering humans when negotiating, they are able to offer issues outside the negotiation domain, to convince the opponent to get this deal. Well known examples are money or, in the case of cars, extra tires. In the current \texttt{Genius} environment it is impossible to offer new issues in a negotiation session, maybe this could be a feature in the future when negotiators are able to rate unknown issues or when they can ask for input of a human. 

\subsection{Feedback concerning genius}

Because our agent's quite complex behavior, we tried to use as much functions
already in the Genius API to reduce our workload. However, using the Genius API 
sometimes was very challenging. Below we make a short list of issues we encountered,
which -- we hope -- can be used to improved the next versions of Genius:

\begin{itemize}
\item It was not possible for our acceptance strategy to communicate with the 
opponent model strategy or opponent model. This is a real problem, 
since for acceptance we wanted to check if bids were pareto optimal and so forth
before accepting. Therefore, we created our \texttt{Helper} class.
\item Conversions between \verb-bidDetails-, \verb-bid- and \verb-bidPoint- classes can be hard. Especially 
converting \verb-bidPoints- back to bid or \verb-bidDetails- seemed impossible, we had to search through
part of the bid-space to find back the bids corresponding to \verb-bidPoints-, as the \texttt{BidPoint.getBid()} method returned \texttt{null}.
\item Built in functions of the opponent's \verb-utilitySpace- do not work for all default included 
opponent models. For example, the Bayesian model does not construct a \verb-utilitySpace- for the opponent. 
In fact, we were not able to construct utility spaces for most models except the hardheaded frequency model. 
\item \verb-Bid-s do not calculate a hash code that gives the same results as the equals function. This would 
be useful for creating a \verb-Dictionary-/\verb-HashMap- where bids are mapped to utilities.
\item It is very hard to retrieve the value names within the issues from the domain (if not impossible).
\item The function to calculate the Kalai Smorodinsky point, which we used intensively,
does not always return the correct bid. This is best demonstrated in the domain 
\verb-NiceOrDie-, where it returns the ``Die'' offers as being the Kalai Smorodinsky point if the opponent model
is not entirely correct, but slightly skewed.
\item In Genius, when running tournament a second time, log files are not correctly saved.
It will try to append the new log file to the old one, and because of this genius will throw
an exception. 
\item In Genius, when loading tournaments or adding agents, the parameters are suddenly scrambled and don't comply to their original order. Also the descriptions are not saved or loaded, so one can not easily interpret the meaning of the parameters anymore. the only fix is to reselect the component, in order to load the descriptions and get the original order back.
\item In Genius, when editing parameters of BOA components, the labels for the parameter names are super small and only display the first few characters. This is especially frustrating with the issue mentioned above.
\item It would be very nice if one could export a whole trace of values from a negotiation
session for analysis and debugging. We have implemented a logging BOA component to do this for us, but this worked very inefficient.
\item When executing a tournament, it would be nice if it was shown in the results table what agent accepted a bid.
\item Columns \texttt{AgentA params} and \texttt{AgentB params} are always empty and all parameters are summarized in the agent names instead, making the former 2 columns obsolete.
\end{itemize}

