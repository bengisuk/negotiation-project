
While an opponent strategy model is not required for a good negotiation agent we have decided to implement a simple version of this component. Our strategy model has one method \texttt{getOpponentModel()} that approximates the behaviour of the opponent. \\

While testing our agent we noticed that \emph{hardheaded} opponents do not offer a large number of different bids, i.e. they stick to bids that have high utility for themselves. On the other hand, agents that act as \emph{conceder} offer a wide variety of bids due to their conceding behaviour. Our strategy model uses this information to determine the strategy of the opponent. \\

Our strategy model first fetches the bid history of the oponent, then these results are filtered on time. Only opponent bids that were offered before $t=0.6$ are taken into account. Then our helper function \texttt{getDistinctBids()} returnes a list of unique bids, i.e. all different bids are returned only once. The number of distinct bids is compared to the total number of opponent bids by calculating the ratio between the two. This ratio is compared to a hardcoded threshold; empirical measurements suggested a threshold of $0.03$ for best results. A ratio below this threshold suggests the opponent to be \emph{hardheaded} since the number of unique bids is small. The method \texttt{getOpponentModel()} then returns that the opponent is hardheaded, while it returns \emph{conceder} otherwise. \\

For our agent, the classification (\emph{hardheaded} vs \emph{conceder}) of the opponent is sufficient. \todo{This section is not yet finished since it is unclear where we are using the OMS...} \\

This is a very simple though efficient method for determining the opponents strategy. For future work we also suggest calculating the \emph{variance} between the opponents bids. A small variance would suggest the agent sticks to its own best utility and can be classified as hardheaded. This approach looks promising, however we did not have enough time to implement is.
