Our bidding strategy is time dependent, we have implemented three phases each of which uses its own strategy for offering bids. The main \texttt{Group7\_BS} class chooses the strategy by evaluating the normalized time (\texttt{getTime()}) and the two hard-coded phase thresholds stored in the array \texttt{phaseBoundary}. In this section we discuss the strategy behind each of the phases separately.

\subsubsection{First Phase}
Our agents first bid is determined by the method \texttt{determineOpeningBid()}. By offering a bid of maximal utility the opponent can easily estimate our preference profile. Due to this reason we have chosen to offer an initial bid of utility $0.9$. The best bid of this utility is obtained by calling \texttt{getBidNearUtility()} of the \texttt{OutcomeSpace} class. \\

After determining our openings bid and receiving the initial bid from the opponent our agent continues by generating random bids. These random bids are generated within a small utility range. The (average) height of this range decreases by a small amount when time passes. The range is calculated by the method \texttt{getBidRange()} given in Listing~\ref{code:getrangefunctionfirstphase}. After calculating the range, the \texttt{getRandomBid()} method is used to generate a bid within this range. \\

We set the parameters such that in the beginning bids are generated around an utility of $1$, that is $[0.98, 1.0]$, and then gradually decreases towards the end. At the end of the phase the bids are centered around an utility of $0.9$ which corresponds to $[0.88, 0.92]$. As these numbers indicate, we set the margin of the bid range to $0.02$. \todo{Update this, we now also look at the best bid of the opponent.}

\begin{lstlisting}[caption=Code for calculating bid range as function of normalized time, label=code:getrangefunctionfirstphase]
public Range getBidRange (double t, double margin) {
  double normTime = t/this.phaseEnd; // Normalized time
  
  // Center of the utility range
  double val = 1-(normTime/10);
  
  // Best bid that the opponent has offered so far
  BidDetails bestOpponent = negotiationSession.getOpponentBidHistory().getBestBidDetails();
  
  // Set the bounds for the range
  double lb = val-margin;
  double ub = val+margin;
  
  if (bestOpponent.getMyUndiscountedUtil() > val){
    // The opponent has offered a better bid (for us)
    // than the center of our bid range, we counter this
    // by choosing a bid higher (+0.05) than the opponents bid.
    val = bestOpponent.getMyUndiscountedUtil()+0.05;
    
    // Update boundaries
    lb = val; ub = val+margin;
  }
  
  // Range in which bids are randomly generated
  Range r = new Range(lb, ub); 
  
  // Set upper bound to 1 if it exceeds upper bound
  if (r.getUpperbound() > 1) r.setUpperbound(1.0);
  
  return r;
}
\end{lstlisting}

\subsubsection{Second Phase}

\subsubsection{Third Phase}

The third phase only consists of the very last part of the negotiation session (usually from $t = 0.95$). The strategy that is employed here relies on the opponent model strategy. Our method \texttt{getOpponentModel()} from the OMS returns whether the opponent is assumed to be \emph{HardHeaded} or \emph{Conceder}. Based on what is returned by the OMS the next bid is determined as described below.

\begin{description}
  \item[Opponent is assumed to be Conceder] \hfill \\
  In this case the next bid that is offered is the Kalai Point.

  \item[Opponent is assumed to be HardHeaded] \hfill \\
  Analyzing the opponents behaviour is...\todo{Fix after Olli changed his things...}
\end{description}

