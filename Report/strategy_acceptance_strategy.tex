% !TEX root = negotiation_final_report.tex
During the development of our acceptance strategy BOA component, $AS_{BOAconstructor}$, we designed, implemented, and tested many different acceptance conditions. Some failed horribly, but some exceeded our expectations. Within this paper we will only discuss the acceptance conditions that made it into the final product that was delivered along with this report.\\

Our acceptance strategy component consists of 7 acceptance conditions that are mainly based on the simple acceptance conditions as found in \cite{baarslag2013acceptance}. However, we modified these basic conditions according to our theories. To the best of our prior knowledge, they should all perform better together than any other acceptance strategy. Further along this report, we will discuss the tests we performed to evaluate our claims.\\

In the following subsections, we will discuss all 7 of our conditions included in the \texttt{determineAcceptability()} method of our $AS_{BOAconstructor}$.

\begin{description}
  \item[AC Curve] \hfill \\
This acceptance condition is based on the simple $AC_{const}$\cite{baarslag2013acceptance}, but we added an extra dimension. $AC_{curve}$ depends on the normalized current time, $t_{current}$, of the negotiation and modifies the constant accordingly. We created a formula that would exponentially decrease when $t_{current}$ is close to $t_{end} = 1.0$ but would show almost non decreasing properties close to the beginning to the negotiation sessions. In the final version of $AC_{curve}$, we let it start at a utility boundary $f_{curve}(0.0) = 1.0$, and approach the first bid of the opponent, $U_{other}^{1}$, so that $f_{curve}(1.0) = (1 - \alpha) * 1.0 + \alpha * U_{other}^{1}$, where $\alpha$ is a percentage. This was preferred over approaching a static value, as it accommodates for varying preference profiles. However, we hereby assume that the opponent will most probably first send his best bid, in order to get his highest utility, namely $U_{opponent}^{1} = 1.0$. Until now we have not seen any agent that disagrees with this assumption, except the Simple Agent agent, which always selects a random bid. \\

The actual formula for the curve, implemented in the \texttt{getAcceptCurveValue()} method, is defined as $f_{curve}(t) = \mu * min(1.0, (1 + \lambda - t^2)^{\phi})$, where $\mu$ is the starting value of the curve, and $\lambda$ is a tiny value that allows the curve to end at a real value and at $f_{curve}(1.0)$ is not an asymptote. Furthermore, $\phi$ is the exponent which is calculated by solving for $f_{curve}(1.0) = (1 - \alpha) * 1.0 + \alpha * U_{other}^{1}$. The values used in our final version are $\alpha = 0.5$, $\mu = 1.0$, $\lambda = 0.005$.

  \item[AC Panic] \hfill \\
Our $AC_{panic}$ condition is based on $AC_{time}$\cite{baarslag2013acceptance}, but instead of looking at a threshold for a normalized time value, we approximate the amount of bids that are possibly left and only accept after less than $\beta$ bids are left. This is done by calculating the time, $\Delta~t_{bid}$, that was needed for each bid, with the help of a running average. The remaining time, $t_{remaining} = 1 - t_{now}$, is divided by the averaged time per bid to get the amount of bids that are probably left: $\#_{bids~left} = round(t_{remaining} / \Delta t_{bid})$. This is implemented in the \texttt{guessBidsLeft()} method of our $AS_{BOAconstructor}$. The value for $\beta$ used in our final version is $\beta = 3$.\\

Furthermore, 2 extra conditions were used before we accept with $AC_{panic}$. Firstly, if the last bid of the opponent was better than the perceived Kalai point minus a conceding value, say $\gamma = 0.05$, we accept for sure, since the received bid will be almost as good as a win-win or even better. Secondly, since we are in 'panic mode', we will also always accept when the opponent offers us a better offer than his best offer yet, minus the same conceding value $\gamma$.

  \item[AC Horror] \hfill \\
This condition is also based on $AC_{time}$, just like our previous condition. However, $AC_{horror}$ will always accept whenever we think that only 1 bid is left for us to propose, $\#_{bids~left} = 1$, as it is always worse to end up with no agreement.

  \item[AC Kalai] \hfill \\
Whenever our utility of the opponent's last bid is really close to the Kalai point, $U_{kalai} - 0.01 <= U_{opponent}^{last} <= U_{kalai} + 0.01$, $AC_{kalai}$ always accepts, as this is a nice win-win situation. We only do this, though, when we are pretty sure that our calculated Kalai point is reliable. We do this by checking whether the opponent has made enough distinct bids to have an accurate enough opponent model.

  \item[AC Nash] \hfill \\
The acceptance condition $AC_{nash}$ is exactly the same as our $AC_{kalai}$, except for the use of the Nash point instead of the Kalai point.

  \item[AC Worst] \hfill \\
This condition is inspired on $AC_{next}$\cite{baarslag2013acceptance}, but instead of accepting when the received bid is higher than our next bid, we accept when the opponent's last bid is higher than our worst bid, minus an time-based increasing conceding value $\delta(t) = t * 0.05$, so that we accept when $U_{opponent}^{last} > U_{our}^{worst} - \delta(t_{now})$. However, to prevent this value from accepting too quickly, we created a time-dependent formula that we use to cut off the said threshold $U_{our}^{worst} - \delta(t_{now})$. The formula is defined as follows, $threshold(t) = max(\epsilon * (1-time) + \zeta, U_{our}^{worst} - \delta(t_{now}))$, where in our final version $\epsilon = 0.3$ and $\zeta = 0.6$. This way, our $AC_{worst}$ will never accept anything below a utility of 0.6, and when $t_{now} = 0.0$, the condition will never accept anything below 0.9.

  \item[AC Next] \hfill \\
The last acceptance condition is simply the $AC_{next}$ as mentioned in \cite{baarslag2013acceptance}.

\end{description}

Finally, our acceptance strategy only looks back within a sliding time window of $\omega = 0.2$, so that the behavior of most of the above mentioned acceptance conditions varies during the course of a negotiation session. This is to accommodate for changing behaviors in our and the opponent's strategies. Thus, we never look at the entire time line of a negotiation session, in order to determine the acceptability of the opponent's last bid.